\documentclass{assignment-263}

\anum{3}
\course{CSC263}
\duedate{Feb 25, 2019}
\filename{ps3.pdf, ps3.tex, pizza.py}

\begin{document}
\think

\begin{enumerate}
	\item \textbf{[14]} In this question, we will study a pitfall of
		quadratic probing in open addressing. In the lecture, we showed
	 for quadratic probing that we need to be careful choosing the
		probe sequence, since it could jump in such a way that some of the
		slots in the hash table are never reached.
		\begin{enumerate}
			\item Suppose that we have an open addressing hash table of size
				$m=7$, and that we are using \textbf{linear} probing of the
				following form.
				\[
					h(k, i) = (h(k) + i)~\text{mod}~m,\quad i = 0, 1, 2\dots
				\]
				where $h(k)$ is some arbitrary hash function. We claim that,
				as long as there is a free slot in this hash table, the
				insertion of a new key (a key that does not exist in the
				table) into the hash table is guaranteed to succeed, i.e.,
				we will be able to find a free slot for the new key. Is this
				claim true? If true, concisely explain why; if not true,
				give a detailed counterexample and justify why it shows that
				the claim is false.

			\item Now, suppose that we have an open addressing hash table
				of size $m=7$, and that we are using \textbf{quadratic}
				probing of the following form.
				\[
					h(k, i) = (h(k) + i^2)~\text{mod}~m,\quad i = 0, 1, 2\dots
				\]
				where $h(k)$ is some arbitrary hash function. We claim again
				that, as long as there is a free slot in this hash table,
				the insertion of a new key into the hash table is guaranteed
				to succeed, i.e., we must be able to find a free slot for
				the new key. Is this claim true? If true, concisely explain
				why; if not true, give a detailed counterexample and justify
				why it shows that the claim is false.

			\item If either of your answers to (a) and (b) is ``false'',
				then it means that some of the slots in the hash table are
				essentially ``wasted'', i.e., they are free with no key
				occupying them, but the new keys to be inserted may not be
				able to use these free slots. In this part, we will show
				an encouraging result for quadratic probing that says ``this
				waste cannot be too bad''.

				Suppose that we have an open addressing hash table whose
				size $m$ is a prime number greater than 3, and that we are
				using \textbf{quadratic} probing of the following form.
				\[
					h(k, i) = (h(k) + i^2)~\text{mod}~m,\quad i = 0, 1, 2\dots
				\]
				Prove that, if the hash table contains less than $\lfloor
				m/2 \rfloor$ keys (i.e., the table is less than half full),
				then the insertion of a new key is guaranteed to be
				successful, i.e., the probing must be able to reach a free
				slot.

				\textbf{Hint:} What if the first $\lfloor m/2\rfloor$ probe
				locations for a given key are all distinct? Try proof by
				contradiction.
		\end{enumerate}

	\newpage
Solution: \vskip5pt
(a) The claim is true. Since h(k) is hash function, it is the index where the key is stored in the given array with size m = 7. So the range of h(k)+i is none negative numbers, which implies that h(k,i) will always in the range of [0,6]. Besides, the key for insertion does not exist in the table, which means there will be no duplicate keys. In conclusion, inserting a new key by using linear probing is guaranteed to be true. \vskip5pt
(b) The claim is false. Counterexample: in the question, m = 7, $h(k,i) = (h(k) + i^2)$. The possible numbers we can get after $i^2$ mod 7 are: \vskip5pt
$1^2$ mod 7 = 1; $2^2$ mod 7 = 4;$3^2$ mod 7 = 2;$4^2$ mod 7 = 2;$5^2$ mod 7 = 4;$6^2$ mod 7 = 1;$7^2$ mod 7 = 0 $=>{1,4,2,2,4,1,0}$. And we cannot reach slots like 3,5 and 6. In conclusion, by using quadratic probing, there will be some slots unreachable in the given array. So it is not guaranteed to succeed to find a free slot for the new key.\vskip5pt
(c) assume $0\le i,j \le m/2$ and $i \neq j$(since all elements in the first half array are distinct).Then we prove by contradiction:\vskip5pt
we assume that $i \neq j$ and $(h(k) + i^2)$ mod (m/2) = $(h(k) + j^2)$ mod (m/2)\vskip5pt
\qquad \qquad \qquad \qquad $=>$ $i^2$ mod (m/2) = $j^2$ mod (m/2)\vskip5pt
\qquad \qquad \qquad \qquad $=>$ ($i^2$ - $j^2$) mod (m/2) = 0 \vskip5pt
\qquad \qquad \qquad \qquad $=>$ $i^2$ - $j^2$ = 0 (basic algebra) \vskip5pt
\qquad \qquad \qquad \qquad $=>$ (i+j)(i-j) = 0 \vskip5pt
The contradiction is false, because we assumed $i \neq j$, so both (i+j) or (i-j) can not be 0. Therefore, if $i \neq j$, then $(h(k) + i^2)$ mod (m/2) $\neq$ $(h(k) + j^2)$ mod (m/2), which also means that if the hash table contains less than m/2 keys the insertion of a new key is guaranteed to be successful,



\item \textbf{[14]} Suppose that we have an array $A[1, 2, \dots]$ (index
	starting at 1) that is sufficiently large, and supports the
	following two operations \texttt{INSERT} and \texttt{PRINT-AND-CUT}
	(where $k$ is a global variable initially set to $0$):
\begin{python}
def INSERT(x):
    k = k + 1
    A[k] = x

def PRINT-AND-CUT():
    for i from 1 to k:
        print A[k]
    k = k // 2     # integer division
\end{python}

		We define the cost of the above two operations as follows:
		\begin{itemize}
			\item The cost of \texttt{INSERT} is exactly $1$.
			\item The cost of \texttt{PRINT-AND-CUT} is exactly the value of
				$k$ before the operation is executed.
		\end{itemize}

		Now consider any sequence of $n$ of the above two operations. Starting
		with $k=0$, perform an amortized analysis using the following two
		techniques.

		\begin{enumerate}
			\item Use the \textbf{aggregate method}: First, describe the
				worst-case sequence that has the largest possible total
				cost, then find the upper-bound on the amortized cost per
				operation by dividing the total cost of the sequence by the
				number of operations in the sequence.

			\item Use the \textbf{accounting method}: Charge each inserted
				element the smallest amount of ``dollars'' such that the
				total amount charged always covers the total cost of the
				operations. The charged amount for each insert will be an
				upper-bound on the amortized cost per operation.
		\end{enumerate}
		\textbf{Note:} Your answer should be in \textbf{exact forms} and
		your upper-bound should be as tight as possible. For example, $7$
		would be a tighter upper-bound than $8$, $\log_2 n$ is tighter than
		$\sqrt{n}$, and $4n$ is tighter than $5n$. Your upper-bound should
		also be a simple term like 7, 8 or $3\log n$, rather than something
		like $(5n^2 - n\log n) / n$. Make sure your answer is clearly
		justified. 

\end{enumerate}

\newpage
Solution: \vskip5pt
(a) assume the size of the given array is n. Then there will be $\log_2n$ times that the PRINT-AND-CUT function will be executed. Therefore, the total cost can be written as: T(n) = n + $\sum_{i=0}^{\log_2(n-1)} 2^i$\vskip5pt
by geometric, we can get that S(n) = $2^0 + 2^1 + 2^2 + ... 2^{\log_2(n-1)}$\vskip5pt
\qquad \qquad \qquad \qquad 2S(n) =  $2^1 + 2^2 + ... 2^{\log_2(n-1)} + 2^{\log_2(n-1)+1}$\vskip5pt
\qquad \qquad \qquad \qquad $=> S(n) = 2^{\log_2(n-1)+1} - 1$\vskip5pt
so T(n) = n + $2^{\log_2(n-1)+1} - 1$\vskip5pt
\qquad $\le n + 2^{\log_2(n-1)+1}$\vskip5pt
\qquad $\le n + 2^{\log_2(n)+1}$\vskip5pt
\qquad $= n + 2n$ = 3n = O(n)\vskip5pt
Total cost is O(n), and we have total number of n operations. Therefore, the average upper-bound = $n/n$ = 1 = O(1)\vskip5pt
(b) Assume the array is empty now. Since the cost of insert is O(1), for every element, we need 1 dollar to insert. And once we insert, we need one dollar to print the element. Then we need 1 dollar extra to help other cases. So the minimum cost will be 3 dollars. Therefore, the total cost will be 3n(O(n)). So the average upper-bound = $n/n$ = 1 = O(1)
\program

\begin{enumerate}
	\item[3.] \textbf{[12]} Dan's favourite food is pizza. (If you haven't
		tried the Cow Pie pizza in DH, you should!)

		Imagine that every pizza in the world is a circle, with exactly five slices.
		For each slice, Dan gives the integer quality rating of the slice. We say
		that two pizzas are equivalent if one pizza can be rotated so that the
		quality of each corresponding slice is the same.

		For example, suppose that we had these two pizzas:
		$(3, 9, 15, 2, 1)$
		and
		$(15, 2, 1, 3, 9)$
		These two pizzas are equivalent: the second is a rotation of the first.

		However, the following two pizzas are {\bf not} equivalent:
		$(3, 9, 15, 2, 1)$
		and
		$(3, 9, 2, 15, 1)$
		because no rotation of one pizza can give you the other.

		Here's another example of two pizzas that are {\bf not} equivalent:
		$(3, 9, 15, 2, 1)$
		and
		$(9, 15, 2, 1, 50)$

		We say that two pizzas are the same {\bf kind} if they are equivalent.

		In Python, a pizza will be represented as a tuple of 5 integers.
		Your task is to write the function \verb|num_pizza_kinds|, which
		determines the \textbf{number} of different kinds of pizzas in the
		list.


Requirements:
\begin{itemize}
\item Your code must be written in Python 3, and the filename must be \verb|pizza.py|.
\item We will grade only the \verb|num_pizza_kinds| function; please do not change its signature in the starter code. include as many helper functions as you wish.
   \item You are {\bf not} allowed to use the built-in Python
			dictionary.
			\item To get full marks, your algorithm must have average-case runtime
				$\mathcal{O}(n)$. You can assume Simple Uniform Random Hashing.
   \end{itemize}

\textbf{Write-up}: in your \verb|ps3.pdf/ps3.tex| files, include the following: an explanation of how your code works, justification of correctness, and
 justification of desired $\mathcal{O}(n)$ average-case runtime.
 
 Justification:\vskip5pt
 The \verb|compare_two_pizza| function takes two tuples in the pizzas, and make comparison between the two tuples. Inside this function, the two for loop take about O(n) runtime each, since they are basically linear search. So the runtime of  \verb|compare_two_pizza| function is O(n). Then in the function \verb|is_pizza_in|,  it determines whether the given tuple is in the given list by using \verb|compare_two_pizza| function. So the runtime should be $O(n^2)$. Lastly,in the \verb|num_pizza_kinds|function, everytime we see a different kind of pizza, we append the pizza into an empty list. Then by calling \verb|is_pizza_in| function, we can count the total number of different kinds of pizzas. So the final runtime will be O($n^2$*n) = O($n^3$). And the total number of operations would be $n^2$, since we are searching the given list of tuples twice to find all possible pair of tuples. Therefore, the average-case runtime will be $n^3 / n^2$ = $n$ = O(n).

 \end{enumerate}

\end{document}